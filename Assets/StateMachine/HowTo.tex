
\includegraphics[width=1.00000in,height=1.00000in]{Pictures/1000000000000200000002001C418E1631B5388B.jpg}

Game State Machine

HowTo

Build a framework for games both simple and complex

© 2016 Todd D. Vance

\includegraphics[width=0.50000in,height=0.27083in]{Pictures/1000020100000780000004383F0E7001620C8600.png}Permission
is hereby granted, free of charge, to any person obtaining a copy of
this software and associated documentation files (the ``Software''), to
deal in the Software without restriction, including without limitation
the rights to use, copy, modify, merge, publish, distribute, sublicense,
and/or sell copies of the Software, and to permit persons to whom the
Software is furnished to do so, subject to the following conditions:

The above copyright notice and this permission notice shall be included
in all copies or substantial portions of the Software.

THE SOFTWARE IS PROVIDED ``AS IS'', WITHOUT WARRANTY OF ANY KIND,
EXPRESS OR IMPLIED, INCLUDING BUT NOT LIMITED TO THE WARRANTIES OF
MERCHANTABILITY, FITNESS FOR A PARTICULAR PURPOSE AND NONINFRINGEMENT.
IN NO EVENT SHALL THE AUTHORS OR COPYRIGHT HOLDERS BE LIABLE FOR ANY
CLAIM, DAMAGES OR OTHER LIABILITY, WHETHER IN AN ACTION OF CONTRACT,
TORT OR OTHERWISE, ARISING FROM, OUT OF OR IN CONNECTION WITH THE
SOFTWARE OR THE USE OR OTHER DEALINGS IN THE SOFTWARE.

\section{}\label{section}

\section[Introduction]{\texorpdfstring{\protect\hypertarget{anchor}{}{}Introduction}{Introduction}}\label{introduction}

A Finite State Machine, more technically called a Deterministic Finite
Automaton is a theoretical concept that models a simple robot that
follows simple rules: it is in a state; it receives an input; Based on
the current state and the input, it does something, and it goes into a
new state (or the same state again). It has been ``invented'' many times
in many different forms over the centuries (Euclid's straightedge and
compass constructions are probably the earliest), but the most common
model used in computer programming, similar to the one used here (we
actually use a variant, called the Mealy model), was first described in
detail in the mid 20\textsuperscript{th} century scholarly paper:
\emph{Moore, Edward F (1956). ``Gedanken-experiments on Sequential
Machines''. Automata Studies, Annals of Math. Studies. Princeton, N.J.:
Princeton University Press (34): 129--153.} (A ``Gedanken-experiment''
is a thought experiment, done in the head instead of in a laboratory).

In real life, finite state machines occur in lots of everyday products,
nowadays controlled by computers but in earlier times, controlled by a
mechanical process. An example would be a typical vending machine.

\subsection[Mouse Trap
Model]{\texorpdfstring{\protect\hypertarget{anchor-1}{}{}Mouse Trap
Model}{Mouse Trap Model}}\label{mouse-trap-model}

\includegraphics[width=6.91667in,height=2.95833in]{Pictures/100002010000073C000003199C6E1582B991AA31.png}A
very simple, mechanical example of a state machine is the standard
mousetrap (not the modern techy humane ones---I don't know how they
work): it has two states, set (as shown in the image), and unset. The
inputs are mouse or no mouse, or ``set the trap''. The behaviors are
spring, or do nothing. If it is in the ``unset'' state, it does nothing,
whether or not there is a mouse, and stays in the unset state. If it is
in the ``set'' state, it springs if a mouse is present and goes to the
unset state. If there is no mouse, it does nothing and remains in the
set state. If it is unset and a human does ``set the trap'', it goes to
the set state. If it is set and the human does ``set the trap'', this is
really an error condition, and what actually happens might be that the
trap springs and it goes to the unset state and the human learns his
lesson.

\subsection[Vending Machine
Model]{\texorpdfstring{\protect\hypertarget{anchor-2}{}{}Vending Machine
Model}{Vending Machine Model}}\label{vending-machine-model}

A typical vending machine spends most of its time in the ``Ready''
state, with lights on and (at least for soft drinks) refrigeration
running.

The possible inputs are: press a product button (of which there are
several), activate the ``coin return'' (though nowadays it returns bills
too), or insert coins (or bills).

When in the ready state, if you press a product button, a typical
machine would show the price of the product on a small display. Note
that what makes a state machine a ``state machine'' is that the behavior
depends not just on the input, but the state it is in. If the machine
were in another state, it might dispense a product instead of just
showing the price when the product button is pressed.

When in the ready state, if you press the coin return, nothing happens.
(If something happens, such as money being returned, it means a previous
customer walked away when the machine was in a different state and lost
their money).

\includegraphics[width=4.00000in,height=5.15625in]{Pictures/10000000000004B00000060E213205DDB5C4A0D2.png}When
in the ready state, if you start inserting money, now something useful
happens. It changes to the ``Waiting for Selection'' state.
(technically, there is additional state information here: the amount of
money inserted so far).

When in the waiting for selection state, if you insert money, the amount
of money inserted increases but nothing else happens.

When in the waiting for selection state, if you press the coin return,
you get your money back (if the machine is working\ldots{}strange how
many are not working\ldots{}) and the machine returns to the Ready
state.

When in the waiting for selection state, if you press a product button,
there are two possibilities: you have inserted enough money for the
product or you didn't. If you inserted enough money, it goes to the
Dispense state. Otherwise, you get some kind of error message and it
stays in the Waiting for Selection state.

When in the dispense state: none of the inputs do anything. It just
dispenses the product, returns any change if any, and goes back to the
Ready state.

Note that this description of the Vending Machine state machine is a bit
simplified. To make it a more accurate model, there would be additional
states like ``enough money'', ``not enough money'', ``return change
state'', and so on. That a state and input can have two possible results
(enough money to buy product or not enough) means more states are needed
to be truly accurate, since this is not allowed with the technical
definition of a pure deterministic automaton. But this should be enough
to illustrate what a state machine is: a simple robot that follows
simple rules.

\subsection[What is a State
Machine]{\texorpdfstring{\protect\hypertarget{anchor-3}{}{}What is a
State Machine}{What is a State Machine}}\label{what-is-a-state-machine}

A state machine has a finite set of ``states'' and a set of possible
``inputs'' and ``behaviors''. The state machine is always in one state.
If it receives an input, then depending on which input it receives and
which state it is in, it will perform one or more behaviors, and then
either stay in the same state or transition to another state. Then it
waits again for input and continues in the same way.

\subsection[Exercise: Draw Your Own State
Machine]{\texorpdfstring{\protect\hypertarget{anchor-4}{}{}Exercise:
Draw Your Own State
Machine}{Exercise: Draw Your Own State Machine}}\label{exercise-draw-your-own-state-machine}

Consider these exercises to be ``additional fun,'' but also, a way of
solidifying knowledge and enhancing future experience with state
machines. Even a good but unsuccessful attempt will benefit the reader.
Some exercises have Google-able solutions (or partial or even complete
solutions may occur later in this document\ldots{}), but the author
suggests making a good try before looking. Remember, there are no
grades---the goal of these exercises is to ``think,'' and to gain some
knowledge as a result.

The first exercise is: think of some system you are familiar with (like
the Vending Machine example) and write up a model of it as a state
machine. Possible ideas include: other kinds of vending machines, an
Automated Teller Machine (ATM), an espresso machine, or even some games
can be modeled as state machines. Note text adventure games are really
large state machines.

\subsection[Exercise: Make it
Precise]{\texorpdfstring{\protect\hypertarget{anchor-5}{}{}Exercise:
Make it
Precise}{Exercise: Make it Precise}}\label{exercise-make-it-precise}

Make your knowledge of state machines more precise. For example, if you
are a mathematician, come up with a precise mathematical definition of
state machine, say, around the level of precision of the definitions in
Euclid's Elements. If you are a programmer, write code that implements a
state machine. For example, you could try to beat Roller Coaster
Simulator or The Sims or Flight Simulator or Sim City and create Vending
Machine Simulator! The prototype would, in your favorite programming
language, take input as entered text or button presses, and behaviors
(and which state the machine is in) would be shown by printing text. If
this is too much, write Mouse Trap Simulator.

\subsection[Exercise: Do Some
Research]{\texorpdfstring{\protect\hypertarget{anchor-6}{}{}Exercise: Do
Some
Research}{Exercise: Do Some Research}}\label{exercise-do-some-research}

Using Google, Wikipedia, or even textbooks you have access to, read up
on state machines. If you can, try to find the paper by Edward F. Moore.
If not, google what you can find on his design (excerpts of the paper
and writings about the paper are there). Find out things like, what
other kinds of state machines there are and how they differ and how many
of them are also essentially the same thing, why the Vending Machine
``breaks the rules'' of a deterministic automaton, and answer to your
own satisfaction the question: ``Is my computer a (really big) state
machine?'' Note there are very good reasons to answer either Yes or No
to that one! Look up Turing Machines (which came before Moore's
paper---Turing's paper I think is available online, full text, and is a
good read, including the reasons he designed the Turing Machine the way
he did---because of this paper, Alan Turing is considered to be the
inventor of computer science) to find some good reasons the answer
should be ``No''. For a bonus, answer to your own satisfaction, ``Is the
physical universe a (really, really big) state machine?''. Note that
Isaac Newton thought the answer to that was, ``Yes,'' (but possibly an
infinite state machine) but modern physicists tend toward ``No''. The
idea is to find out why (Newton and modern physicists both have good
reasons for their choices). And even though Newton was religious, the
reasons don't even consider the religious element, which makes the
question more complex\ldots{}. Many of these ideas would make good term
papers for certain courses---check with your teacher.

\section[Game State
Machine]{\texorpdfstring{\protect\hypertarget{anchor-7}{}{}Game State
Machine}{Game State Machine}}\label{game-state-machine}

The particular state machine this paper is about is the overall game
state machine. This doesn't have much to do with the details of game
play, but of the organization of levels, cutscenes, title screens, and
so on in a game. Let us begin with an example.

\subsection[Classic Arcade State
Machine]{\texorpdfstring{\protect\hypertarget{anchor-8}{}{}Classic
Arcade State
Machine}{Classic Arcade State Machine}}\label{classic-arcade-state-machine}

\includegraphics[width=4.00000in,height=5.15625in]{Pictures/10000201000004B00000060E1317354F3039A2BE.png}Consider
the arcade game Pac-Man, the kind that was in a wooden box at an arcade
that one stood in front of and played after having fed it quarters. We
shall discuss the state machine for it, noting that many classic arcade
games had state machines that followed a similar pattern. When you
walked up to the game, it was most likely in ``attract'' mode, the mode
that tries to get you to spend a quarter. This is a big state that will
be divided into smaller states (title, demo, and so on---note
Namco/Bally probably have their own names for the states; I'm making up
names that are reasonably descriptive). But before the attract mode,
there is another mode not seen often unless you pull the plug and plug
it back in: booting up.

\begin{enumerate}
\def\labelenumi{\arabic{enumi}.}
\tightlist
\item
  \textbf{Boot}: this is the initial state, the start state of the game.
  It takes a certain amount of time to finish, and when it does, it
  immediately transitions to the Title state. Input is not noticed
  (although, for all I know, there might be special codes to do certain
  tests, etc. that the arcade operator could do at this point).
\item
  \textbf{Title}: the screen that shows the title of the game and some
  graphics to make someone come closer for a better look. After a
  certain amount of time has passed, it goes to the Description state.
  If a quarter is spent and the Play button is pressed, it instead goes
  to the Start Game state.
\item
  \textbf{Description}: here is a screen describing game play and points
  for various activities. After a certain amount of time has passed, it
  goes to the High Scores state (I might have Description and High
  Scores in the wrong order; it has been decades since I've played
  Pac-Man at an arcade). If a quarter is spent and the Play button is
  pressed, it instead goes to the Start Game state.
\item
  \textbf{High Scores}: the top ten scores, with initials, are displayed
  here. After a certain amount of time has passed, it goes to the Demo
  state. If a quarter is spent and the Play button is pressed, it
  instead goes to the Start Game state.
\item
  \textbf{Demo}: A sample game is shown, without sounds and possibility
  for player input. It doesn't last long before Pac-Man is eaten and the
  game returns to the Title state. If a quarter is spent and the Play
  button is pressed, it instead goes to the Start Game state.
\item
  \textbf{Start Game}: the Level 1 board is displayed, the intro music
  is played, Pac Man and the ghosts appear on the board, and then the
  game transitions to the Play Game state. Input is ignored during the
  Start Game state.
\item
  \textbf{Play Game}: now the input is connected to the game and a user
  can play. Depending on the events of the game, the next state is Level
  Up or Game Over (recall, this is just overall game state, we are not
  modeling the fine points of game play at his level).
\item
  \textbf{Level Up}: when the board is cleared, this state is entered.
  The board flashes, and there may be a cutscene. When this is done, the
  next level is loaded and the game goes back to Play Game.
\item
  \textbf{Game Over}: when the player runs out of lives, the game is
  over. A big Game Over message appears over top the game board while
  the ghosts dance in glee at having earned another quarter. After a
  time, one of two things happen. Either it returns directly to the
  Title state, or if the score achieved is in the top ten, it goes to
  the New High Score state.
\item
  \textbf{New High Score}: the player is given the opportunity to enter
  his initials to appear on the high score board. After the initials are
  entered (or if it times out with it stuck on AAA, since some players
  just walk away), the game returns to the High Score state--or rather,
  a duplicate of the High Score state (I'll call it High Scores Reprise)
  that transitions to the Title rather than to the Demo.
\item
  \textbf{High Scores Reprise}: the top ten scores, with initials, are
  displayed here, with the just-made high score highlighted. After a
  certain amount of time has passed, it goes to the Title state. If a
  quarter is spent and the Play button is pressed, it instead goes to
  the Start Game state.
\end{enumerate}

Note that one can ``factor'' this state machine. The outer state machine
has for major states: Boot, Attract, Play, and Post Game. Within each
major state is a refinement into more states.

Note also that because the Pac-Man state machine is such that each major
state can be entered from another major state at only one inner state
(namely, Boot, Title, Start Game, and New High Score, respectively), one
could actually program this with 5 state machines: the outermost major
state machine with 4 major states, and within each major state, start up
a minor state machine (the Boot one is trivial, just one state) for the
inner minor states. Even if it is not set up so that each major state
can be entered at only one inner state, it could be done with some
additional code. In real life programming, state machines are often not
``pure'' but require additional code for special cases.

\subsection[Exercise: Draw Your Own Game State
Machine]{\texorpdfstring{\protect\hypertarget{anchor-9}{}{}Exercise:
Draw Your Own Game State
Machine}{Exercise: Draw Your Own Game State Machine}}\label{exercise-draw-your-own-game-state-machine}

Pick a computer game, whether arcade, mobile, desktop, console, or even
hand-held (like GameBoy or the old single-game machines like Football),
and draw a game state diagram. Remember, this is the top level state
machine for the game and doesn't include finer game play details.

\subsection[Exercise: Code the Framework of
Pac-Man]{\texorpdfstring{\protect\hypertarget{anchor-10}{}{}Exercise:
Code the Framework of
Pac-Man}{Exercise: Code the Framework of Pac-Man}}\label{exercise-code-the-framework-of-pac-man}

Code up, in your favorite language, the Pac-Man state machine. Either
use the full 11-state machine or factor it into 5 state machines, the
outermost one being the controlling one having 4 states. Keep it simple,
accepting input like ``play game'' and printing out how it responds and
what state it is now in. This code could then be used as a framework for
the full game.

\subsection[Exercise: Code the Framework of Any
Game]{\texorpdfstring{\protect\hypertarget{anchor-11}{}{}Exercise: Code
the Framework of Any
Game}{Exercise: Code the Framework of Any Game}}\label{exercise-code-the-framework-of-any-game}

For the state machine you made in Exercise 2.2, code up a simulator in
your favorite programming language. Keep it simple, accepting input like
``play game'' and printing out how it responds and what state it is now
in. This code could then be used as a framework for the full game.

\section[Coding a State Machine in Unity3d
C\#]{\texorpdfstring{\protect\hypertarget{anchor-12}{}{}Coding a State
Machine in Unity3d
C\#}{Coding a State Machine in Unity3d C\#}}\label{coding-a-state-machine-in-unity3d-c}

\href{https://unity3d.com/}{Unity3d} is a popular, general purpose game
engine that is free to use for hobbyists, though some premium pay
features are also available. The
\href{https://www.udemy.com/unitycourse/learn/v4/}{online Udemy course}
is not free, but is written for Unity Version 4; with a Version 5 course
in the works (to be sold at a discount to those who purchased Version
4). In addition, the \href{https://unity3d.com/}{Unity3d} site has some
tutorials and manuals and a forum for learning Unity.

A basic state machine can be coded as a C\# class called StateMachine
whose main responsibility is to keep track of which state the machine is
in. A state is then an instance of another C\# class, called State,
which maintains its name and a list of states it can transition to. The
simplest possible state machine code would thus look something like
this, in a single file called
\href{https://gist.github.com/tdvance/15aee1cfccecc49c2ce51c038b064120}{SimpleStateMachine.cs}.
This code is specific to Unity3d, but it could easily be adapted to
other systems and languages. In Unity, one can fill in state machine
details using the Inspector. In other languages, it might be necessary
to write code to fill in the details of a specific state machine. This
version uses strings to label states, as well as a string array to list
states that can be transitioned to, making it easy to fill in using
plain text. In addition to the previous link, the state machine code is
reproduced in the first Appendix{[}Page
\protect\hyperlink{anchor-13}{14}{]}.

To use the simple state machine in Unity:

\begin{enumerate}
\def\labelenumi{\arabic{enumi}.}
\tightlist
\item
  Create an empty GameObject in the Hierarchy, and rename it to
  StateMachine
\item
  Attach the SimpleStateMachine.cs script to the game object as a
  component
\item
  Select the StateMachine game object just created in the Hierarchy and
  look at it in the Inspector
\item
  Add new states at will, and transitions out of each state to other
  states.
\item
  Run the game. Note how the state can be changed by changing the value
  of the currentState field in the SimpleStateMachine script component.
\item
  Write an additional script that calls the Transition method of the
  SimpleStateMachine.cs script with an integer argument to select which
  state to transition to.
\end{enumerate}

There are several problems with this simple state machine. First of all,
all it does is show what state it is in. There is no additional behavior
(though that can easily be added) and no means for transitioning in
response to events. The second is a small issue since one only need put
the StateMachine method Transition into an event (such as a button
click) to cause the transition to happen upon that event.

A more serious problem is it is currently impossible to load a new scene
without losing the state machine instance. This means that to be an
overall game state machine, the state machine should be a Singleton
class. That means, any game has only one state machine (other state
machine classes could be used for smaller state machines that don't
persist between scenes) and it remains active across scene changes.

\subsection[Exercise: Fix the State Machine
Class]{\texorpdfstring{\protect\hypertarget{anchor-14}{}{}Exercise: Fix
the State Machine
Class}{Exercise: Fix the State Machine Class}}\label{exercise-fix-the-state-machine-class}

At a minimum, think of how you would do this, Googling for ``Unity
DontDestroyOnLoad'' to find out how to make a Singleton. If able, modify
the code to make it robust and useful and test it well. As a bonus, redo
some of the previous exercises with this new code. As an extended bonus,
actually create a complete, playable game using this new code. The
reader may wish to compare what they wrote with the
\href{https://github.com/tdvance/GameStateMachine}{author's State
Machine code}.

\subsection[``Boss Level'' Exercise: Design and Build a Better State
Machine Asset for Unity
3D.]{\texorpdfstring{\protect\hypertarget{anchor-15}{}{}``Boss Level''
Exercise: Design and Build a Better State Machine Asset for Unity
3D.}{Boss Level Exercise: Design and Build a Better State Machine Asset for Unity 3D.}}\label{boss-level-exercise-design-and-build-a-better-state-machine-asset-for-unity-3d.}

Start with a list of features it should have (for example, automatically
transition after a certain amount of time has elapsed, or---this is very
advanced---create a graphical Editor interface to build the state
machine). Prioritize them into important, very useful, nice to have, and
optional.

Using the priorities as a guide, built a state machine asset having as
many of those features as you are able to do, understanding that what
looks easy at the design stage might turn out to be hard when you
actually do it.

Test the result. Fix any bugs you find. Are there non-bugs that are
still annoyances? Are there things that could be made to work better?
Are there new features that appear useful in retrospect not in the
original list? Amend the design and create a better version.

Test again. Be paranoid and test for anything that could possibly go
wrong. Give it the acid test: build some games using the state machine
and notice what goes wrong or what slows you down and think of how it
can be improved.

Test again.

When all is working magnificently, go to the
\href{https://unity3d.com/}{Unity3d} website and research how to submit
an asset to the \href{https://www.assetstore.unity3d.com/en/}{Asset
Store}---the procedure is not trivial! But it can be done. The author
has done it. Good luck and hope it gets accepted.

Also look at comparable assets in the store. If yours is significantly
better than the others available, consider asking for money! That
requires even more work and set up. If it isn't, can you amend the
design and improve it? Some of the ones there have some advanced,
hard-to-code features.

\section[Using the GameStateMachine
asset]{\texorpdfstring{\protect\hypertarget{anchor-16}{}{}Using the
GameStateMachine
asset}{Using the GameStateMachine asset}}\label{using-the-gamestatemachine-asset}

The \href{https://github.com/tdvance/GameStateMachine}{author's State
Machine code} can be used as an overall state machine. The
StateMachine.cs class is found in the Assets/StateMachine folder. In
addition are Unity3d scenes for testing: ClassicArcadeStart and
MobileStyleStart. The first is similar (but not exactly like) the
Pac-Man example above. The second is a style similar to what might be
found on a mobile game like Angry Birds. To use these, all scenes in
subfolders must be in the Build Settings. The two scenes mentioned give
two examples of how to set up the state machine for a game. The key
object is the StateMachine prefab (in the Assets/StateMachine folder).
Put this in the start scene of a game. It is recommended the start scene
be a special scene that only loads Singletons (like the StateMachine)
and is never returned to again.

Use the inspector to add new states and fill in details, such as
transitions, auto transitions (after a certain number of seconds),
scenes to load, and so on. Three events are defined for each state:
entering, exiting, and loading the scene. The DefaultActions.cs file is
a simple class that simply logs the events. A game programmer will
likely provide their own classes for actions to be invoked upon events
happening.

\subsection[Exercise: Set up the GameStateMachine project in
Unity3D]{\texorpdfstring{\protect\hypertarget{anchor-17}{}{}Exercise:
Set up the GameStateMachine project in
Unity3D}{Exercise: Set up the GameStateMachine project in Unity3D}}\label{exercise-set-up-the-gamestatemachine-project-in-unity3d}

Download the entire folder structure from
\href{https://github.com/tdvance/GameStateMachine}{GitHub}\href{https://github.com/tdvance/GameStateMachine}{from
the link given}. Also ensure you have Unity version 5.5 or higher
(though it should still be Unity 5). Then open the base folder (which
contains the Assets directory and other files) in Unity. It should
import everything. Make sure all scenes are in the Build Settings (In
the project window's header bar, there is a ``Search By Type'' icon that
looks like a little circle, square, and triangle. Click this and select
``Scene'' from the dropdown menu to show all scenes. Select all, drag to
Build Settings (under the File menu)). Test-run the ClassicArcadeStart
and the MobileStyleStart scenes and make sure all the possible button
combinations work. Read through the StateMachine in the Inspector and
try to understand how it works.

\subsection[Exercise: Build a Simple GameStateMachine in
Unity3D]{\texorpdfstring{\protect\hypertarget{anchor-18}{}{}Exercise:
Build a Simple GameStateMachine in
Unity3D}{Exercise: Build a Simple GameStateMachine in Unity3D}}\label{exercise-build-a-simple-gamestatemachine-in-unity3d}

Now, start with a new scene. Put the StateMachine prefab in it and build
either the Vending Machine simulator or the Mouse Trap Simulator that
was done in a previous exercise{[}Page
\protect\hyperlink{anchor-5}{6}{]}. Use UI buttons for simulating the
various inputs and maybe print to the log (with the Debug.Log(string)
method) to show states and behaviors.

\subsection[Exercise: Build the Pac-Man GameStateMachine in
Unity3D]{\texorpdfstring{\protect\hypertarget{anchor-19}{}{}Exercise:
Build the Pac-Man GameStateMachine in
Unity3D}{Exercise: Build the Pac-Man GameStateMachine in Unity3D}}\label{exercise-build-the-pac-man-gamestatemachine-in-unity3d}

Again, start with a new scene and build the Pac-Man system described
above. Compare this with the ClassicArcadeStart state machine.

\subsection[Exercise: Build a
Game]{\texorpdfstring{\protect\hypertarget{anchor-20}{}{}Exercise: Build
a Game}{Exercise: Build a Game}}\label{exercise-build-a-game}

Create a complete, playable game of your choice using the
GameStateMachine assets.

\subsection[Exercise: Improve \textbf{the} GameStateMachine
assets]{\texorpdfstring{\protect\hypertarget{anchor-21}{}{}Exercise:
Improve \textbf{the} GameStateMachine
assets}{Exercise: Improve the GameStateMachine assets}}\label{exercise-improve-the-gamestatemachine-assets}

The author left plenty of room for improvement. What can you improve?
Redo the exercise{[}\protect\hyperlink{anchor-15}{11}{]} from the last
section using this asset package.

\section[Appendix: Simple State Machine
code]{\texorpdfstring{\protect\hypertarget{anchor-13}{}{}Appendix:
Simple State Machine
code}{Appendix: Simple State Machine code}}\label{appendix-simple-state-machine-code}

\href{https://gist.github.com/tdvance/15aee1cfccecc49c2ce51c038b064120}{SimpleStateMachine.cs}
file is shown here:

using System.Collections;

using System.Collections.Generic;

using UnityEngine;

using System;

/// \textless{}summary\textgreater{}

/// Maintain the current state that the state machine is in

/// \textless{}/summary\textgreater{}

public class SimpleStateMachine : MonoBehaviour \{

\#region Public interface

/// \textless{}summary\textgreater{}

/// A state in the state machine

/// \textless{}/summary\textgreater{}

{[}Serializable{]}

public struct State \{

{[}Tooltip(``Name of the State''){]}

public string name;

{[}Tooltip(``List of states this state can transition to''){]}

public string{[}{]} canTransitionTo;

//constructor for convenience: needed only to create a default ``Start''
state.

public State(string name) \{

this.name = name;

\}

\}

{[}Tooltip(``Changing this changes the state the machine is in''){]}

public string currentState = startState;

{[}Tooltip(``All the available states''){]}

public State{[}{]} states = \{ new State(startState) \};

public void Transition(int which) \{

currentState = activeState.canTransitionTo{[}which{]};

\}

\#endregion

//the name of the start state, what the machine is in upon awakening

private static string startState = ``Start'';

//the currently active state

State activeState;

// Use this for initialization

void Start() \{

//set the current state

activeState = FindState(currentState);

\}

// Update is called once per frame

void Update() \{

if (currentState != activeState.name) \{

//if state changes, reset the current state

activeState = FindState(currentState);

\}

\}

//find state having given name

State FindState(string name) \{

foreach (State state in states) \{

if (state.name == name) \{

return state;

\}

\}

Debug.LogError(``Missing state: '' + name);

return null;

\}

\}

\section[Appendix: Mathematical Definition of a Deterministic Finite
Automaton]{\texorpdfstring{\protect\hypertarget{anchor-22}{}{}Appendix:
Mathematical Definition of a Deterministic Finite
Automaton}{Appendix: Mathematical Definition of a Deterministic Finite Automaton}}\label{appendix-mathematical-definition-of-a-deterministic-finite-automaton}

The key object in a DFA is a set, namely the state set S. Because it is
a finite automaton, S must be finite. Other than that, there is no real
restriction for what is in the set S. The elements can be thought of as
labels for the states. In the vending machine example, S = \{Ready,
Waiting, Dispensing\}, a set with three elements.

To make the definition precise, we need a second set, I, the set of
inputs. For the vending machine example, I = \{InsertMoney,
PressCoinReturn, and PressProductButton\}. So we have two sets so far.

In addition to the set, the mathematical definition must model the
transitions, and this is best done with a function. A function in
mathematics is a little different from a function in computer
programming. Given two sets S and T, we say a function f maps elements
of S to elements of T if for any element s we choose from S, we have
f(s) = t for a \emph{unique} element of T. For example, let S = \{Ready,
Waiting, Dispensing\} and let T (temporarily) be the set of positive
integers \{1, 2, 3, 4, 5, \ldots{}\}. Let f(s) be defined as ``the
number of letters in s''. Then f is a function from S to T: f(Ready) =
5, f(Waiting) = 7, and f(Dispensing) = 10. It satisfies the requirements
of a function: it maps each element of S to a unique element of T. There
are 3 elements of S, and each one is mapped to something in T. And it is
not the case that some element is mapped to two different values in T.
Note that it is \emph{not} required that all of T be used in the map.
For example, 1 is not the result of mapping any element of S by the
function f.

But the transition function is a different type of function from what
was just described. It takes two inputs. So we modify the function a
bit. If S and I are sets, and T is a set, then a two-argument function f
maps a pair (s,i) consisting of an element s of S and an element i of I,
to a unique element t of T.

For an example, again take S = \{Ready, Waiting, Dispensing\} and I =
\{InsertMoney, PressCoinReturn, and PressProductButton\}, and take T to
be the set of positive integers again: \{1, 2, 3, 4, 5, \ldots{}\}. Let
us define a two-argument function f whose first argument is from S and
whose second argument is from I, and which sends these pairs of values
to elements of T. Suppose f(s,i) is defined to be ``the number of
letters in s times the number of letters in i''. Then,
f(Ready,InsertMoney) = 5*1 = 55. f(Dispensing, PressProductButton) = 10
* 18 = 180. In fact, for each pair, the first item from S and the second
item from I, f produces a unique element of T. Note that again not all
elements of T are used. It's also ok for a function to produce the same
element of T from different inputs.

Now we can specify the transition function. The transition function is a
function t having two arguments, a state from S and an input from I. The
result of the transition function is a state from S: it could be the
same state as was input, or a different state.

Thus, for the vending machine example, take S = \{Ready, Waiting,
Dispensing\} and I = \{InsertMoney, PressCoinReturn, and
PressProductButton\}, and the transition function t is as follows. Note
there are a few issues: I mentioned before the vending machine example
isn't technically correct!

\begin{itemize}
\tightlist
\item
  t(Ready, InsertMoney) = Waiting
\item
  t(Waiting, InsertMoney) = Waiting
\item
  t(Dispensing, InsertMoney) = ??? (here is a place the vending machine
  model is incorrect; let is make the answer Waiting for this case to
  fix it)
\item
  t(Ready, PressCoinReturn) = Ready
\item
  t(Waiting, PressCoinReturn) = Ready
\item
  t(Dispensing, PressCoinReturn) = Ready (this is another place the
  original model was broken.)
\item
  t(Ready, PressProductButton) = Ready
\item
  t(Waiting, PressProductButton) = Dispensing (this is where the
  original model was really broken. We need states to take into
  consideration enough or not enough money!)
\item
  t(Dispensing, PressProductButton) = Ready (another place the original
  model was inaccurate).
\end{itemize}

Finally, there are outputs, which we called ``behaviors'' above. The
outputs of a DFA are a finite set O. In the case of the vending machine,
O = \{AdjustInsertedAmount, ReturnMoneyAndResetInsertedAmount
DispenseProductAndReturnMoneyAndResetInsertedAmount, ShowPrice,
Nothing\}

The output map is kind of like the transition map: it is a function of
two arguments, the first being a state and the second being an input,
and it maps the pair (s,i) with s a state and I an input, to an output o
in the set O. For the vending machine example, the output map looks like
this (with some fixes already applied):

\begin{itemize}
\tightlist
\item
  o(Ready, InsertMoney) = AdjustInsertedAmount
\item
  o(Waiting, InsertMoney) = AdjustInsertedAmount
\item
  o(Dispensing, InsertMoney) =
  DispenseProductAndReturnMoneyAndResetInsertedAmount
\item
  o(Ready, PressCoinReturn) = Nothing
\item
  o(Waiting, PressCoinReturn) = ReturnMoneyAndResetInsertedAmount
\item
  o(Dispensing, PressCoinReturn) =
  DispenseProductAndReturnMoneyAndResetInsertedAmount
\item
  o(Ready, PressProductButton) = ShowPrice
\item
  o(Waiting, PressProductButton) = Nothing
\item
  o(Dispensing, PressProductButton) =
  DispenseProductAndReturnMoneyAndResetInsertedAmount
\end{itemize}

To make the physical machine fit the mathematical description, it was
necessary to make some of the behaviors ``big'' and do several things.
To design this in the real world, one would actually use more states.
Actually, it would not be a pure state machine (under the mathematical
description) because the money counter and the possibility of different
products having different prices means it is better to make some of the
transitions depend on the result of executing code that checks money
received against product prices. So in reality, a vending machine is
\emph{almost} a state machine. I believe some of the old vending
machines used code written in COBOL. There are a lot of devices out
there that still run COBOL! However, languages like Java are replacing
it, slowly.

So, putting it all together:

Definition(Discrete Finite Automaton): A DFA, or Finite State Machine
(of the Mealy type), consists of three finite sets S, I, and O, whose
elements are called, respectively, the states, the inputs, and the
outputs (or sometimes behaviors), and two functions: a transition
function t of two arguments mapping ordered pairs from S and I to
elements of S, and an output (or behavior) function o of two arguments
mapping ordered pairs from S and I to elements of O.

\subsection[Exercise: Other Definitions of State
Machines]{\texorpdfstring{\protect\hypertarget{anchor-23}{}{}Exercise:
Other Definitions of State
Machines}{Exercise: Other Definitions of State Machines}}\label{exercise-other-definitions-of-state-machines}

Do research and find out about Moore-type state machines,
non-deterministic state machines, stochastic (or probabilistic) state
machines, and Turing machines, and write precise mathematical
definitions for these.

For more ``points'' (that you'll never actually see!) don't just copy
other definitions, but learn what they mean well enough to write them
down without looking.

For a bonus, look not only at Turing's original machine, variants like
the pushdown automaton, the multi-tape machine, and so on.

To really solidify your knowledge, code up simulators for these kinds of
machines.

To really, really solidify your knowledge, build games using what you
coded.
